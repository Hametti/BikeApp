	\newpage
\section{Testowanie}	%5
%Opisujemy testy, sprawdzamy czy nie generuje błędów.
\subsection{Menu} %5.1
Aby testowanie było bezproblemowe i kompleksowe należy przetestować każdą część aplikacji stopniowo zwiększając stopień skomplikowania. Pierwszym testem będzie działanie menu, składa się on z następującego scenariusza:\\\\
1)Otworzenie aplikacji\\
2)Sprawdzenie czy strona główna to widok zaimplementowany w klasie MainPage\\
3)Sprawdzenie czy każda opcja w menu pokrywa się z zaimplementowanym~widokiem\\
4)Ustawienie aplikacji na działanie w tle, ponowne włączenie, i sprawdzenie czy otwiera się na ostatniej przeglądanej stronie\\
5)Przełączanie zakładek używając multitoucha\\\\
Wyniki: Aplikacja działa poprawnie, nie pojawia się żaden bug\\\\
\newpage
\subsection{Śledzenie trasy} %5.2
Scenariusz:\\
1)Otwarcie zakładki 'Start Tracking'\\
2)Wciśnięcie przycisku 'Start Tracking'\\
3)Sprawdzenie czy tekst na przycisku zmienił się na 'Stop Tracking'\\
4)Wciśnięcie przycisku 'Stop tracking' przed upływem 10 sekund od wciśnięcia 'Start Tracking'\\
5)Sprawdzenie czy przycisk spowrotem zmienił się na 'Start Tracking' oraz czy aplikacja nie zaproponowała zapisania trasy\\
6)Ponowne wciśnięcie przycisku 'Start Tracking' i odczekanie 10 sekund\\
7)Wciśnięcie przycisku 'Stop Tracking'\\
8)Sprawdzenie czy aplikacja zaproponowała zapisanie trasy\\
9)Zapisanie trasy\\
10)Sprawdzenie czy w zakładce 'My routes' została dodana trasa\\
11)Sprawdzenie czy trasa posiada poprawne wartości GPS z pozycji kodu\\\\
Wyniki: Wystąpił bug. Śledzenie trasy działa, ale pokazuje na mapie jedynie linię prostą z początku i końca śledzenia.\\\\
\newpage
\subsection{Wyświetlanie tras} %5.3
Scenariusz:\\
1)Otwarcie zakładki 'Your routes'\\
2)Sprawdzenie czy wcześniej dodane trasy poprawnie się wyświetlają\\
3)Sprawdzenie czy po kliknięciu na trasę wyświetlają się jej szczegóły oraz opcja jej usunięcia\\
4)Kliknięcie 'Display on map' oraz sprawdzenie czy wyświetla się poprawna trasa\\
5)Kliknięcie 'Delete', przełączenie się spowrotem na zakładkę 'Your routes' i sprawdzenie czy trasa się usunęła\\\\
Wyniki: Wystąpił bug. Trasa się usuwa, ale lista aktualizuje się dopiero po wyjściu z zakładki 'Your routes' oraz ponownym wejściu.
\subsection{Mapa} %5.4
Scenariusz:\\
1)Otwarcie zakładki 'Map'\\
2)Sprawdzenie mapa poprawnie się wyświetla\\
3)Kliknięcie opcji 'Update Location'\\
4)Sprawdzenie czy lokalizacja została uaktualniona, i czy aplikacja wycentrowała mapę na aktualnej lokalizacji\\
5)Sprawdzenie czy działa zoom za pomocą przycisków z ikonami + i -, oraz za pomocą gestu\\
6)Sprawdzenie czy działa uaktualnianie lokalizacji za pomocą ikony w prawym górnym rogu \\\\
Wyniki: Aplikacja działa poprawnie, nie pojawia się żaden bug\\\\
\newpage
\subsection{Pomiar rzeczywisty} %5.5
Scenariusz:\\
1)Otwarcie zakładki 'Live measurements'\\
2)Włączenie akcelerometra za pomocą slidera\\
3)Sprawdzenie czy po odpowiednim obrocie telefonu wartości się poprawnie wyświetlają. Wartość powinna wynosić zero jeśli telefon jest ustawiany prostopadle do siły grawitacji względem odczytywanej osi\\
4)Podrzucenie telefonu i sprawdzenie czy zmieniła się wartość maksymalnego zanotowanego przeciążenia G\\
5)Włączenie funkcji 'Shake Detection'\\
6)Kilkukrotne potrząśnięcie telefonem i sprawdzenie czy pojawiło się okno z powiadomieniem 'Shake detected'\\\\
Wyniki: Aplikacja działa poprawnie, nie pojawia się żaden bug.

\subsection{Statystyki} %5.6
Scenariusz:\\
1)Dodanie za pomocą kodu tras z tak policzonymi wartościami aby dało się porównać ich statystyki liczone za pomocą aplikacji z wynikiem rzeczywistym\\
2)Otwarcie zakładki 'Statistics'\\
3)Sprawdzenie czy wszystkie pola tekstowe wyświetlają poprawne wartości(maksymalnie 2 miejsca po przecinku, poprawny format czasowy)\\
4)Porównanie wyników pokazywanych przez aplikacje z rzeczywistymi obliczeniami średnich na podstawie dodanych tras\\\\
Wyniki: Aplikacja działa poprawnie, nie pojawił się żaden bug\\\\

\newpage
\subsection{Ustawienia} %5.7
Scenariusz:\\
1)Uruchomienie aplikacji, sprawdzenie czy domyślnie jest włączony tryb ciemny\\
2)Otwarcie zakładki 'Settings'\\
3)Wyłączenie trybu ciemnego za pomocą slidera\\
4)Otwarcie po kolei każdej z zakładek i sprawdzenie czy zarówno tło jak i tekst wyświetlają się poprawnie w trybie jasnym\\
5)Otwarcie zakładki 'Settings'\\
6)Ponowne włączenie trybu ciemnego za pomocą slidera\\
7)Przejście po kolei do każdej z zakładek i sprawdzenie czy zarówno tło jak i tekst wyświetlają się poprawnie w trybie ciemnym\\\\
Wyniki: Aplikacja działa poprawnie, nie pojawił się żaden bug\\\\

